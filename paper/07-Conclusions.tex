In this paper we investigated the application of PAC search to BitTorrent tracking data. The study was motivated by an increasing awareness of the security issues inherent to the currently used BitTorrent protocols. PAC search offered a solution to the security issues but introduced an unknown drop in efficiency. PAC search's unstructured network design solves problems of censorship and information deletion. It does so at the expense of bandwidth costs, storage costs and information retrieval accuracies. This paper set out to determine the extent to which these negative factors would prohibit a PAC search solution over BitTorrent. We conclude that PAC search over BitTorrent tracking data is, generally, feasible. The models and strategies presented here outline a PAC solution capable of retrieving 70\% of all observed torrents after a single search. Under normal working conditions current BitTorrent protocols can offer 100\% retrieval accuracy. Under adversarial working conditions these same, existing, protocols can be reduced to 0\% accuracy. The PAC search protocol offers a practical solution resistant to failures of this magnitude.

This paper does not look at the effect of node drop out on the performance of the system. Future work needs to be done to account for the likely lowering of search accuracy under network churn. Furthermore, some interesting work could be done looking at the effect of dynamically changing $z$. There would seem to be some efficiency increases possible if nodes start querying a small number of nodes and then increase $z$ after an unsuccessful search.