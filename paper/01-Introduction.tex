%!TEX root = paper.tex
BitTorrent is a popular method of distributing multimedia content, software and other data. BitTorrent requires users to interact with a centralised service called the tracker. This central focus point for the protocol is a potential security weakness that could be exploited to disrupt the network. Studies have shown tracker failure to be a common and disruptive occurrence \cite{Pouwelse2005}. Disrupting the tracker's service effectively halts the running of the BitTorrent network. In order to strengthen the network to attack, a distributed hash table (DHT) extension has been introduced and widely adopted. The DHT spreads the responsibilities of the tracker across the network and thereby makes it more difficult to disrupt the tracking service. However, whilst this improves the security of BitTorrent, the DHT mechanism is also vulnerable to attack. For example, \cite{Timpanaro2011,Sit2002} show that the most popular DHT implementation for BitTorrent allows malicious nodes to passively monitor nodes and even remove nodes from the network.

Information discovery or retrieval in an unstructured P2P network is more resistant to attacks attempting to censor that information \cite{Lua2005}. However, since it is not practical to search the entire network, the accuracy of such search is both probabilistic and approximate. Recent work on modelling probably approximately correct (PAC) search has provided a strong mathematical framework for modelling the search accuracy, i.e. the probability of finding a torrent, which is primarily a function of the number of nodes queried and the number of nodes a torrent is replicated onto \cite{Asthana2012,Cox,Cox2009}.

To determine whether PAC search in feasible on the BitTorrent network, we first conducted a 64 day study of BitTorrent activities, looking at the distribution of 1.6 million torrents on 5.4 million peers. Our measurements show that the torrent distribution across nodes follows a power-law. The vast majority of torrents are known to very few nodes. Section~\ref{sec:measurement} describes this work. Given the current distribution of torrents, a probabilistic search is unlikely to succeed.

We say that a node is participating in a torrent if it is downloading or uploading the torrent's data. Currently, nodes in a BitTorrent network know only of the torrents that they are participating in. In order for a PAC search to be successful, a search query must reach at least one node that is participating in the torrent searched for. Unfortunately, finding a participating node is difficult, as Section~\ref{sec:measurement} reveals. To improve the probability of a successful search, Section~\ref{sec:extension} proposes a modification to the BitTorrent protocol such that, if a node receives a query for a torrent it is not participating in, it stores the torrent's ID, together with the address of the querying node.  If the queried node then receives a subsequent query for this torrent, it responds with the address of the previous querying node(s). This modification substantially improves the probability of a successful search for a torrent in the network. Section~\ref{sec:extension} provides a detailed analysis.

Section~\ref{sec:overheads} then considers the overheads associated with the modification of the protocol. Additional bandwidth is required in order to discover torrents and each node must provide a small amount of local storage for indexing purposes. We show that even under extreme conditions the overheads introduced by this extension are not prohibitive. Search queries cost between 6.8KB and 8.8KB and our extension requires only 3.5Kbytes of local storage per hour.

In order to verify our model we run a series of extensive simulations in Section~\ref{sec:simulations}. These simulations confirm our theoretical analysis. The simulations also consider various models of network churn in order to demonstrate the extension's effectiveness in real-world environments.
