%!TEX root = paper.tex
The security weaknesses of the BitTorrent protocol are well known. Improvements to the protocol have been made to alleviate this issue. However, even the DHT-based extensions have proven susceptible to attack.  Since unstructured networks are usually more resistant to attack, this paper investigated the feasibility of a probabilistic (PAC) search to discover torrents.  

The performance of PAC search is strongly dependent of the number of nodes queried and the distribution of torrents in the network. A two month study of the distribution of torrents across nodes showed a power law distribution that is not amenable to PAC search. To address this issue we proposed a modification of the BitTorrent protocol such that each node in the network now indexes a random subset of tracking data. Each node's local database is independently constructed by recording the torrent ID, i.e. infohash, together with the IP address of the querying node. A subsequent analysis of the distribution of tracking data revealed that the tracking data is replicated sufficiently to support a PAC search. Moreover, the communication and storage overheads associated with the modified protocol were shown to be small. Thus, no degradation in performance of BitTorrent is expected.

Simulations were performed on a network of 5 million nodes under a variety of torrent query rates and churn rates. The simulation results support our theoretical analysis.
  
We envision that PAC search could be used to complement rather than replace existing torrent discovery mechanisms. There are a number of directions for future work. These include (i) developing a mechanism to adaptively select the number of nodes queried based on the popularity of the queried torrent, and (ii) developing a mechanism for nodes participating in a torrent to adaptively issue dummy queries so that a torrent's tracking data is sufficiently replicated to guarantee that the probability of a successful search is high.