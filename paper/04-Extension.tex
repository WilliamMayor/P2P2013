Until now, our analysis has assumed that in order to find a torrent and begin the download process a node must directly identify any node currently participating in that torrent. This need not be the case. Rather, to find a torrent, we could first discover the torrent's tracking data. The tracking data will contain a list of nodes thought to be participating in the torrent, these nodes can be used to initiate the download. This is, of course, analogous to the original BitTorrent protocol's torrent-discovery-via-trackers mechanism. To accomplish this in a unstructured P2P environment, we need a mechanism by which each peer indexes a random, non-disjoint subset of torrent tracking data. Given this mechanism, we can apply PAC search on the collection of tracking data, where each torrent's tracking data is equivalent to a document. The success of PAC search is then dependent on the distribution of torrent tracking data, rather than the distribution of the torrents themselves. 

In Section~\ref{sec:extension:indexing} we first introduce the modification to the BitTorrent protocol that enables peers to index a random, non-disjoint subset of torrent tracking data. Section~\ref{sec:extension:model} provides a mathematical model of our extension. Section~\ref{sec:extension:discussion} then analyses the distribution of torrent tracking data and the associated performance of PAC search.

\subsection{Indexing}\label{sec:extension:indexing}

    The ability of a PAC search system to return successful results is influenced by these two, controllable, factors: (i) the distribution of documents across the network, $r_i$, and (ii) the number of nodes contacted per query, $z$. The distribution of torrents over nodes discovered in Section~\ref{sec:measurement} prevents the vast majority of torrents from being discovered no matter how large $z$. Thus, as well as modifying $z$, we must also alter the distribution of the tracking data, $r_i$. From an information retrieval perspective, we do not need to replicate the file, or any pieces of the file, to more nodes. Rather, we need more nodes to be aware of where the torrent can be downloaded from. We need nodes in the network to store the torrent's unique ID and the list of peers participating in the torrent (tracking data), {\em even if the node itself is not participating in the torrent}. To this end, we introduce an index at each node that contains details of previously received requests. When a node is queried for a torrent that it does not own, the queried node stores details of the request in its index, i.e. storing the torrent's unique ID alongside the details of the node that made the request. If the queried node receives any future requests for that torrent, it can now respond with details of a node that it believes to own the torrent. When the querying node receives such a response, it can the contact these nodes directly. Thus, when a node performs a {\em search} it issues one or more queries for a torrent, until such time as a query is successful. A {\em query} consists of a node sending a request to $z$ randomly sampled nodes in the network. Each repeated query for the same torrent selects $z$ different nodes. A successful query is one that results in at least one successful request. A {\em request} consists of a querying node sending the desired torrent's infohash to one of the $z$ random nodes. The queried node responds with either a list of nodes it believes are participating in the torrent, or an empty list. A successful request returns a list of actively participating nodes; nodes that are online and that are downloading or uploading the torrent's data. A unsuccessful request returns either an empty list, or a list of nodes that are not participating in the torrent, e.g. because they have left the network.

\subsection{Model}\label{sec:extension:model}

    When querying a fixed number of random nodes, $z$, the probability of a successful query is now determined by (i) the number of nodes participating in the torrent, and (ii) the number of nodes indexing the torrent's tracking data, $r_i$. Given our proposed modification to the BitTorrent protocol, the more queries the network receives for a torrent, the more that torrent's tracking data is replicated, and the easier it becomes to find. At any point in time the number of nodes indexing a torrent, $r(t)$, additionally depends on: the number, $u$, of requests made for the torrent, and the proportion, $c$, of nodes that have left the network. The change in replication over time can be expressed as:

    \begin{equation}
        {dr(t) \over dt} = u(1+z(1-{r(t) \over n}))-cr(t)
        \label{eq:drdt}
    \end{equation}

    Here, $(1-{r(t) \over n})$ gives the proportion of the $z$ nodes that were not already indexing the torrent. We can solve this ODE to give us an equation for the replication as a function of time:

    \begin{equation}
        r(t) = ke^{-t(c+{uz \over n})}+{un(1+z) \over uz+cn}
        \label{eq:rt}
    \end{equation}    

    The constant, $k$, is given by the initial condition, $r(0)$. Conceptually, this value is the number of nodes that index the torrent before any queries have been made for it. $r(0)$ can be decided by the torrent's authoring node and represents a bootstrap value that enables early queries to be successful. For a fixed $z$ the value for $r(0)$ can be tuned to enable a desired probability of success using Eqn~\ref{eq:prob_find_non_uniform_distribution}. The authoring node simply makes dummy requests to $r(0)$ nodes in order to push tracking data into the network.

    NOTES.

    \begin{enumerate}
        \item With this constant $u$ and $c$ the replication exponentially tends towards ${un(1+z) \over uz+cn}$
        \item I have removed the assumption that nodes continue to query until they are successful. This makes the equations simpler. It removes the safety of knowing that a query definitely increases the replication because it is very possible that a single query is not successful and therefore those requests have not increased tracking data, merely lots of red herrings.
        \item We can add this assumption back in by: increasing the query rate according to the probability of success at $t$, or fixing $P(d)$. I'm more inclined towards the former. EIther might make the ODE equation too hard to solve again.
        \item Replication increases to the limit if $r(0) < {un(1+z) \over uz+cn}$ and decreases if $r(0) > {un(1+z) \over uz+cn}$
    \end{enumerate}

\subsection{Discussion}\label{sec:extension:discussion}

    We saw in Section~\ref{sec:} that because of the distribution of torrents over nodes, PAC search would only be successful for X\% of the queries made. Our BitTorrent extension replicates torrent tracking data in order to improve PAC search performance. In order to analyse the performance gain consider a BitTorrent network in a steady state. Where the number of queries for any torrent is constant over time and the proportion of nodes that leave the network (the churn) is such that total network size remains constant also. We assume that the popularity distribution measured in Section~\ref{sec:measurement} is proportional to the distribution of queries made for the torrent (a reasonable assumption in this steady state). The performance of PAC search over time is then dependent on the behaviour of the replication of the tracking data for the torrent searched for. If the replication increases over time then the torrent will always be discoverable and a worst-case can be guaranteed. If the replication is decreasing over time then eventually the torrent will become undiscoverable. From Eqns~\ref{eq:drdt,eq:rt} we know that...

